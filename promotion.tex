\documentclass[11pt,a4paper]{article}
\usepackage[a4paper, total={16cm, 24cm}]{geometry}

\usepackage{float}  %for better figure placement
\usepackage{placeins}  %for \FloatBarrier

% Basic figure setup, for now with no caption control since it's done
% automatically by Pandoc (which extracts ![](path) syntax from Markdown).
\usepackage{graphicx}


% We will generate all images so they have a width \maxwidth. This means
% that they will get their normal width if they fit onto the page, but
% are scaled down if they would overflow the margins.
\makeatletter
\def\maxwidth{\ifdim\Gin@nat@width>\linewidth\linewidth
\else\Gin@nat@width\fi}
\makeatother
\let\Oldincludegraphics\includegraphics
% Set max figure width to be 80% of text width, for now hardcoded.
\renewcommand{\includegraphics}[1]{\Oldincludegraphics[width=.8\maxwidth]{#1}}
% Ensure that by default, figures have no caption (until we provide a
% proper Figure object with a Caption API and a way to capture that
% in the conversion process - todo).
\usepackage[margin=2cm]{caption}
\captionsetup{format=hang}   %for better captions of figures

\usepackage{adjustbox} % Used to constrain images to a maximum size 
\usepackage{xcolor} % Allow colors to be defined
\usepackage{enumerate} % Needed for markdown enumerations to work
\usepackage{geometry} % Used to adjust the document margins
\usepackage{amsmath} % Equations
\usepackage{amssymb} % Equations
\usepackage{textcomp} % defines textquotesingle
% Hack from http://tex.stackexchange.com/a/47451/13684:
\AtBeginDocument{%
    \def\PYZsq{\textquotesingle}% Upright quotes in Pygmentized code
}

\renewcommand{\familydefault}{\sfdefault}
%\usepackage{palatino}% Nicer default font than Computer Modern for most use cases
\usepackage{helvet}
\usepackage[utf8]{inputenc} %\usepackage[utf8x]{inputenc} problems with bibtex % Allow utf-8 characters in the tex document
% in the case that utf8x and bibtex are to e used simulatnously
%http://tex.stackexchange.com/questions/213173/ucs-and-biblatex-incompatibility-mathletters-and-utf-8
\usepackage[T1]{fontenc}
\usepackage{textcomp}

\usepackage{upquote} % Upright quotes for verbatim code
\usepackage{eurosym} % defines \euro
%\usepackage[mathletters]{ucs} % Extended unicode (utf-8) support   %problems with bibtex
% in the case that mathletters and bibtext are to be used simulatnously_
% http://tex.stackexchange.com/questions/213173/ucs-and-biblatex-incompatibility-mathletters-and-utf-8
\usepackage{fancyvrb} % verbatim replacement that allows latex
\usepackage{grffile} % extends the file name processing of package graphics 
                     % to support a larger range 
% The hyperref package gives us a pdf with properly built
% internal navigation ('pdf bookmarks' for the table of contents,
% internal cross-reference links, web links for URLs, etc.)
\usepackage{hyperref}

\hypersetup{
      pdfauthor = {Thomas Meschede, Thomas.Meschede@tu-berlin.de},
       pdftitle = {TODO: title},
       pdfsubject = {TODO},
       pdfkeywords = {ACS, AOCS, automated design, TODO},
       pdfcreator = {LaTeX with hyperref package},
       pdfproducer = {pdflatex}
}

\input{scripts/pygments}
\usepackage{tabularx}

\usepackage[ngerman, english]{babel}

%%%%%%%%%%%%%%%%%%%%%%%%%%%%%%%%%%%%%%%%%%%%%%%%%%%%%%%%%%%%%%%%%%%%%%%%%%%%%%%%
%%%%text, section and title formating
%%%%%%%%%%%%%%%%%%%%%%%%%%%%%%%%%%%%%%%%%%%%%%%%%%%%%%%%%%%%%%%%%%%%%%%%%%%%%%%%
%paragraph layout
%\setlength{\parskip}{\baselineskip}%
\setlength{\parskip}{5pt}%
\setlength{\parindent}{0pt}%

\colorlet{sectionc}{blue!50!black} 
\colorlet{titlec}{blue!50!black} 

%\renewcommand{\thesubsection}{\hspace*{-12pt}}
\makeatletter  %%move section to the left in the following lines
\let\LaTeX@startsection\@startsection
\renewcommand\section{\@startsection {section}{1}{\z@}%
	{-3.5ex \@plus -1ex \@minus -.2ex}%
	{2.3ex \@plus.2ex}%
	{\normalfont\Large\bfseries\color{sectionc}}}

%%%% remove sections numbers of first level headers
%\def\@seccntformat#1{\csname #1ignore\expandafter\endcsname\csname the#1\endcsname\quad}
%\let\sectionignore\@gobbletwo
%\let\latex@numberline\numberline
%\def\numberline#1{\if\relax#1\relax\else\latex@numberline{#1}\fi}
%\makeatother

\usepackage{blindtext}%random blind text

%%%%%%%%%%%%%%%%%%%%%%%%%%%%%%%%%%%%%%%%%%%%%%%%%%%%%%%%%%%%%%%%%%%%%%%%%%%%%%%%
%%%%%%%configure bibliography
%\iffalse %TODO:  something is wrong here
\usepackage[backend=bibtex,maxalphanames=3, minalphanames=1,style=alphabetic,giveninits,url = false]{biblatex}

\renewcommand*{\labelalphaothers}{}

\DeclareLabelalphaTemplate{
  \labelelement{
    \field[final]{shorthand}
    \field{label}
    \field[strwidth=3,strside=left]{labelname}
  }
  \labelelement{
    \field[strwidth=2,strside=right]{year}    
  }
}

%\bibliographystyle{ieeetr}
%\bibliography{/home/tom/Documents/promotion.bib}
%\addbibresource{promotion.bib}
\addbibresource{/home/tom/ownCloud/bibtex/promotion.bib}
%http://tex.stackexchange.com/questions/204291/bibtex-latex-compiling
%\fi 
%weitere bibtex-Bibliotheken mit {1.bibliothek,2.bibliothek}  hinzufügen
%%%%%%%%%%%%%%%%%%%%%%%%%%%%%%%%%%%%%%%%%%%%%%%%%%%%%%%%%%%%%%%%%%%%%%%%%%%%%%%%



%\usepackage[author={Thomas Meschede}]{pdfcomment}  %for pdf comments

\graphicspath{{/home/tom/ownCloud/iboss/Bilder/}{scripts/}{/home/tom/Dropbox/promotion/scripts}}

\hyphenation{Lageregelungs-konzepte Client-satelliten Lagere-gelungs-systeme Lagere-gelungs-subsystem La-ge-re-ge-lungs-systemen vir-tu-elles Lagere-ge-lung}




%%%%%%%%%%%%%%%%%%%%math
\usepackage[amssymb]{SIunits} %sudo apt install texlive-science
\usepackage{acronym}
%%%%%%%%%%%%%end math

\usepackage{xcolor,colortbl}

%%%%%%%%pagebreak after each section
\let\oldsection\section
%\renewcommand\section{\newpage\oldsection}
\renewcommand\section{\clearpage\oldsection}

\begin{document}
\pagenumbering{roman} % Start roman numbering
\selectlanguage{english}%ngerman

%\subject{Dissertation}
\title{Draft: Untersuchung eines rekonfigurierbaren Lageregelungssystems 
für modulare Raumfahrzeuge}
%\subtitle{}
\author{Autor: Thomas Meschede\\
		Email: Thomas.Meschede@tu-berlin.de}
\date{\today}
%\affiliation{{*}Department of Astronautics, TU Berlin, Germany\\
%e-mail: Thomas.Meschede@tu-berlin.de\\%\vspace{2mm}\\
\maketitle
\normalsize 
\begin{center}
\textbf{Draft: Promotion}
\end{center}
\begin{center}\begin{tabular}{ll}
 Betreuender Professor:  & Prof. Dr. Klaus Brieß\\
            & Institut für Luft- und Raumfahrt, Fachgebiet Raumfahrttechnik\\
            & Technische Universität Berlin
            \end{tabular}\end{center}

\thispagestyle{empty}

\tableofcontents
\section*{Acronyms}
\begin{acronym}
\acro{ACS}{Attitude Control Subsystem}
\acro{AOCS}{Attitude and Orbit Control Subsystem}
\acro{A/D}{Analogue to Digital Converter}
\acro{ADS}{Attitude Determination Subsystem}
\acro{BCRS}{Barycentric Celestical Reference System}
\acro{CAN}{Controller Area Network}
\acro{COG}{Center of Gravity}
\acro{COM}{Center of Mass}
\acro{COS}{Center of Satellite} 
\acro{GCRS}{Geocentric Celestial Reference System}
\acro{DDS}{Distributed Data Service}
\acro{DIA}{Device Identification Address}
\acro{DCO}{Device Coordinates and Orientation}
\acro{ECI}{Earth-Centered Inertial Coordinate System}
\acro{ECEF}{Earth Centered Earth Fixed}
\acro{EPS}{Electrical Power Subsystem}
\acro{FDD}{Feature Driven Development}
\acro{GCRF}{Geocentric Celestial Reference Frame}
\acro{GPIO}{General Purpose Input/Output}
\acro{iBoss}{intelligent Building Blocks for On-Orbit Servicing}
\acro{ICRF}{International Celestial Reference Frame}
\acro{ICRS}{International Celestial Reference System}
\acro{IERS}{Internationl Earth Rotation and Reference Systems Service}
\acro{IMU}{Inertial Measurement Unit}
\acro{ITRS}{International Terrestrial Reference System (}
\acro{ISSI}{Intelligent Space System Interface}
\acro{OCS}{Orbit Control System}
\acro{TEME}{True Equator Mean Equinox ECI}
\acro{TRIAD}{TODO}
\acro{TTC}{Telemetry Tracking and Command}
\acro{QoS}{Quality of Service}
\acro{ROS}{Robot Operating System}
\acro{ROSS}{ROS for Space Applications}
\acro{RTSP}{Rapid Tree Spanning Protocol}
\acro{RTPS}{Real-Time Publish-Subscribe}
\acro{SBF}{Spacecraft Body Frame}
\acro{TCS}{Thermal Control Subsystem}
\acro{MIDC}{Module id coordinates}
\acro{MBF}{Module Body Frame}
\acro{OBC}{On Board Computer}
\acro{OBDH}{On Board Data Handling}
\acro{UART}{Universal Asynchronous Receiver/Transmitter}
\acro{VTi}{Virtual Testbed iBoss}
\end{acronym}

\clearpage  %instead of \newpage %(within chapters)
\pagenumbering{arabic} % Switch to normal numbers
%actual content:
\input{introduction}
\section{Fundamentals of Attitude dynamics and Control}
\input{stateoftheart}
%TODO: vielleicht "\input{}" benutzen (falls pagebreak nicht erwünscht usw..)
\input{./scripts/acs_analysis}
\input{./scripts/autodesign}
\input{./scripts/acssoftware}
\section{Conclusion}

\section*{Appendix}
\addcontentsline{toc}{section}{Appendix}

\section*{TODO}
\label{TODO}
%\clearpage

%\FloatBarrier
%\tiny
%\renewcommand*{\bibfont}{\small}
\printbibliography
%\bibliography{/home/tom/Documents/promotion.bib}

\end{document}
